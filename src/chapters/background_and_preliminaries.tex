\chapter{Background \& {Preliminaries}}
\label{ch:background}

\iffalse
This chapter provides an overview of the mathematical background and
preliminaries needed to understand the rest of the thesis. The chapter is
structured as follows: \autoref{sec:bp:so3_se3} and \autoref{sec:bp:se3} introduce the
special orthogonal group in three dimensions, $\SO$, and the special Euclidean
group in three dimensions, $\SE$, respectively. These groups are essential for
modeling the kinematics and dynamics of \gls{uvms}s. \autoref{sec:pseudoinverse} introduces the
Moore-Penrose pseudo-inverse and the null-space projection, which are extensively used
in the \gls{tpc} framework. \autoref{sec:lagrange} presents Lagrange's equations of motion, which are used to derive the equations of motion for the \gls{uvms}s. \autoref{sec:runge-kutta} introduces Runge-Kutta methods, which are used to numerically solve the equations of motion. Finally, \autoref{sec:background:quaternions} introduces the quaternion representation of rotations, which is used to represent the orientation of the \gls{uvms}s in this thesis. The quaternion representation is also used
in the definition of tasks and the lower-levl \gls{dp} controller presented in
\autoref{sec:tpc:low_level_control}
The contents of this section, as well as all previous sections
are heavily used in the implementation of the controllers and simulator presented in this thesis.
\fi

This chapter provides a comprehensive overview of the mathematical foundations and preliminaries necessary for understanding the remainder of this thesis. The chapter is organized as follows: \autoref{sec:bp:so3_se3} and \autoref{sec:bp:se3} introduce the Special Orthogonal group in three dimensions, $\SO$, and the Special Euclidean group in three dimensions, $\SE$, respectively. These groups are fundamental for modeling the kinematics and dynamics of \gls{uvms}s.
\autoref{sec:pseudoinverse} presents the Moore-Penrose pseudo-inverse and null-space projection techniques, which are extensively employed within the \gls{tpc} framework. \autoref{sec:lagrange} discusses Lagrange's equations of motion, which form the basis for deriving the dynamic equations of \gls{uvms}s. \autoref{sec:runge-kutta} introduces the Runge-Kutta family of methods, which are used for the numerical integration of these equations of motion.
Finally, \autoref{sec:background:quaternions} introduces the quaternion representation of rotations, which is utilized to describe the orientation of \gls{uvms}s throughout this thesis. The quaternion representation also plays a central role in task definition and in the design of the low-level \gls{dp} controller discussed in \autoref{sec:tpc:low_level_control}.

The mathematical concepts introduced in this chapter are important for the implementation of the control strategies and simulation environment developed in this work.


% -----------------------------------------------------------------------------
\section{Special Orthogonal Group in 3 dimensions}
\label{sec:bp:so3_se3}

The special orthogonal group in three dimensions, $\SO$, and the special
Euclidean group in three dimensions, $\SE$, are key mathematical tools in the
study of rigid body dynamics. These groups serve distinct purposes: $\SO$ is
used to represent rotations, while $\SE$ describes rigid body and spatial
transformations. This section offers a brief introduction to these groups and
highlights some of their essential properties.

We denote by $\SO$ the special orthogonal group in three dimensions, which is
the group of all $3 \times 3$ rotation matrices with determinant equal to one,
such that
\begin{align}
    \SO = \{ \bm{R} \in \mathbb{R}^{3 \times 3} \mid \bm{R}^T \bm{R} = \mathbb{I}, \det(\bm{R}) = 1 \}.
\end{align}
Intuitively, $\bm{R}^T \bm{R} = \mathbb{I}$ means that the columns of $\bm{R}$ are orthonormal,
and $\det(\bm{R}) = 1$ means that $\bm{R}$ is a proper rotation, i.e., it does not include
a reflection. The $\SO$ group is closely related to the Lie algebra $\so$
\begin{subequations}
\begin{align}
    \so &= \{ \bm{\Omega} \in \R^{3 \times 3} | \bm{\Omega}^T = -\bm{\Omega} \} \\
    [\bm{\Omega_1}, \bm{\Omega_2}]_{\so} &= \bm{\Omega_1} \bm{\Omega_2}
        - \bm{\Omega_2} \bm{\Omega_1},
\end{align}
\end{subequations}
by the exponential map
\begin{subequations}
\begin{align}
    \exp &: \so \to \SO \\
    \exp &: \bm{\Omega} \mapsto \sum_{k=0}^{\infty} \frac{1}{k!} \bm{\Omega}^k.
\end{align}
\end{subequations}
This is an important property, as it is fundamental to the definition of the
angular velocity $\bm{\omega}$ of a rigid body. Consider a rotation matrix
$\bm{R}_b^i$ that rotates a vector from the body-fixed frame $b$ to some inertial
frame $i$. The body-fixed angular velocity $\bm{\omega}_{ib}^b$ is then defined
as
\begin{align}
    \cross{\bm{\omega}_{ib}^b} = \bm{R}_i^b \dot{\bm{R}}_b^i.
\end{align}
When specifying reference frames, we say that
\begin{align}
    \bm{p}^b = \bm{R}_{a}^b \bm{p}^a,
\end{align}
meaning that the rotation matrix $\bm{R}_{a}^b$ rotates a vector
$\bm{p}^a\in\R^3$ from frame $a$ to frame $b$. Some important properties
of $\bm{R}_a^b \in \SO$ are as follows \cite{modsim}:
\begin{itemize}
\item $\bm{R}_a^b = (\bm{R}_b^a)^{-1} = (\bm{R}_b^a)^T$
\item $\bm{R}_a^b \bm{R}_b^c = \bm{R}_a^c$.
\end{itemize}

Rotations in three-dimensional space can be parameterized using various methods,
including Euler angles, quaternions, angle-axis representations, and
Rodrigues parameters.
In this thesis, quaternions are used to represent rotations. It is however
useful to understand the Euler angle representation, as tasks
are often expressed in terms of Euler angles. This is because the equality
constraint imposed on the quaternion representation makes it less convenient
to define tasks in terms of quaternions.
Euler angles are a set of three angles that describe a sequence of three
rotations about the principal axes of a reference frame. These rotations are
performed in a specific order: typically about the $x$-axis, followed by the $y$
-axis, and finally the $z$-axis. The resulting transformation is represented by
a rotation matrix $\bm{R}$, which encodes the cumulative effect of these three
successive rotations.
Explicitly, if the three Euler angles are denoted as $\phi$, $\theta$, and $\psi$,
corresponding to rotations about the $x$-, $y$-, and $z$-axes respectively,
the rotation matrix is given by \cite{modsim}:
\begin{align}
    \bm{R} = \bm{R}_z(\psi) \bm{R}_y(\theta) \bm{R}_x(\phi),
\end{align}
where
\begin{subequations}
\label{eq:bp:so3:rotations}
\begin{align}
    \bm{R}_x(\phi) &= \begin{bmatrix}
        1 & 0 & 0 \\
        0 & \cos\phi & -\sin\phi \\
        0 & \sin\phi & \cos\phi
    \end{bmatrix}, \\
    \bm{R}_y(\theta) &= \begin{bmatrix}
        \cos\theta & 0 & \sin\theta \\
        0 & 1 & 0 \\
        -\sin\theta & 0 & \cos\theta
    \end{bmatrix}, \\
    \bm{R}_z(\psi) &= \begin{bmatrix}
        \cos\psi & -\sin\psi & 0 \\
        \sin\psi & \cos\psi & 0 \\
        0 & 0 & 1
    \end{bmatrix}.
\end{align}
\end{subequations}
A disadvantage of using Euler angles is the well known problem of gimbal lock.
This occurs when two of the three axes align, causing the rotation matrix to
lose a degree of freedom. We say that the system has lost a degree of freedom
when this happens. In practice it is often possible to avoid this issue by not
controlling the system to a $\pm90^\circ$ pitch angle, or by switching to a
different angle representation when the system approaches gimbal lock. In this
work the issue is avoided by using a quaternion-based representation of the
rotation, which does not suffer from gimbal lock. Quaternions are presented
in \autoref{sec:background:quaternions}.

% -----------------------------------------------------------------------------
\section{Special Euclidean Group in 3 dimensions}
\label{sec:bp:se3}

The special Euclidean group in three dimensions, $\SE$, is the group of all
rigid body transformations in three-dimensional space. It is defined as
\begin{align}
    \SE := \left\{ \begin{bmatrix}
        \bm{R} & \bm{p} \\
        \bm{0} & 1
    \end{bmatrix}
        \in \R^{4 \times 4} \middle| \bm{R} \in \SO, \, \bm{p} \in \R^3
    \right\}.
\end{align}
A transformation by 
\begin{align}
    \bm{T}_b^i = \begin{bmatrix}
        \bm{R}_{b}^i & \bm{p}_{ib}^i \\
        \bm{0} & 1
    \end{bmatrix}
\end{align}
will operate on a vector $\bm{p}_{bo}^b \in \R^3$:
\begin{align}
    \begin{bmatrix}
        \bm{p}_{io}^i  \\
        1
    \end{bmatrix}
    =
    \bm{T}_b^i \begin{bmatrix}
        \bm{p}_{bo}^b \\
        1
    \end{bmatrix}
    =
    \begin{bmatrix}
        \bm{R}_{b}^i \bm{p}_{bo}^b + \bm{p}_{ib}^i \\
        1
    \end{bmatrix}.
    \label{eq:se3_transformation}
\end{align}
One can intuitively see from \autoref{eq:se3_transformation} that the transformation
$\bm{T}_b^i$
takes a vector $\bm{p}_{bo}^b$, the vector from the body-origin $b$ to the point $o$
in the $b$ frame, and transforms it to the vector $\bm{p}_{io}^i$, the vector from
the inertial-frame origin $i$ to the point $o$ described in the $i$ frame.


Consider an element $\bm{H}_{b}^i \in \SE$, which is a transformation from frame
$b$ to frame $i$. The body-fixed twist $\bm{\nu}_{ib}^b$ and the inertial-frame
twist $\bm{\nu}_{ib}^i$ are defined respectively as
\begin{subequations}
\begin{align}
    [\bm{\nu}_{ib}^b]_{\wedge} &= \left(\bm{H}_{b}^i\right)^{-1}\dot{\bm{H}}_{b}^i \label{eq:body_twist_def}\\
    [\bm{\nu}_{ib}^i]_{\wedge} &= \dot{\bm{H}}_{b}^i \left(\bm{H}_{b}^i\right)^{-1}. 
\end{align}
\end{subequations}

Similar to $\SO$, the Lie algebra $\se$ is closely related to $\SE$ by the exponential
map. The Lie algebra $\se$ is defined as
\begin{subequations}
    \label{eq:se3_algebra}
\begin{align}
    \se &= \left\{ \begin{bmatrix}
        \bm{\Omega} & \bm{v} \\
        \bm{0} & 0
    \end{bmatrix}
    \in \R^{4 \times 4}
    \middle|
    \bm{\Omega} \in \so, \, \bm{v} \in \R^3
    \right\} \\
    \left[ \Gamma_1, \Gamma_2 \right]_{\se} &= \Gamma_1 \Gamma_2 - \Gamma_2 \Gamma_1 \\
    \exp &: \se \to \SE
\end{align}
\end{subequations}
We define the matrix representation of the ad-operator
\begin{align}
    \ad_{\se} : \R^6 \to \R^{6 \times 6},
\end{align}
implicitly by requiring that for all $\bm{y} \in \R^6$
\begin{align}
    \ad_{\se}( \bm{x} ) \bm{y} = \bm{x} \wedge \bm{y}. \label{eq:ad_impl}
\end{align}
The resulting $\ad_{\se}(\bm{x})$ matrix is then
\begin{align}
    \ad_{\se} : \begin{bmatrix} \bm{v} \\ \bm{\omega} \end{bmatrix}
        \mapsto \begin{bmatrix}
            [\bm{\omega}]_{\times} & [\bm{v}]_{\times} \\
            \bm{0} & [\bm{\omega}]_{\times}
        \end{bmatrix}.
\end{align}
In a similar fashion, the matrix representation of the $\Ad$ operator
\begin{align}
    \Ad_{\SE} : \SE \to \R^{6 \times 6}
\end{align}
is defined. This operator is implicitly defined by requiring that for all
$\bm{\nu} \in R^6$, $\bm{H} \in \SE$
\begin{align}
    [\Ad_{\SE}(\bm{H})\bm{\nu}]_{\wedge} = \bm{H} [\bm{\nu}]_{\wedge} \bm{H}^{-1}.
\end{align}
This is essentially a mapping from body-fixed twists to inertial-frame twists,
and is extensively used when modeling UVMs. The $\Ad$-operator is then
\begin{align}
    \Ad_{\SE} : \begin{bmatrix}
        \bm{R} & \bm{p} \\
        \bm{0} & 1
    \end{bmatrix}
    \mapsto
    \begin{bmatrix}
        \bm{R} & [\bm{p}]_{\times} \bm{R} \\
        \bm{0} & \bm{R}
    \end{bmatrix}.
\end{align}
The $\Ad$ operator of the inverse of an element of $\SE$ is also widely used
\begin{subequations}
\begin{align}
    \Ad_{\SE}^{-1} &: \begin{bmatrix}
        \bm{R} & \bm{p} \\
        \bm{0} & 1
    \end{bmatrix}
    \mapsto
    \begin{bmatrix}
        \bm{R}^T & - \bm{R}^T [\bm{p}]_{\times} \\
        \bm{0} & \bm{R}^T
    \end{bmatrix} \\
    \Ad_{\SE}^{-1}(\bm{H}) &= \Ad_{\SE}(\bm{H}^{-1}).
\end{align}
\end{subequations}
In the following chapters, the $\ad$- and $\Ad$-operators will always refer to
the $\se$ and $\SE$ Lie algebras and groups, respectively.

% -----------------------------------------------------------------------------
\section{Pseudo-Inverse and Null Space Projections}
\label{sec:pseudoinverse}

For a given matrix $\bm{A} \in \mathbb{R}^{n\times m}$ we define the Moore-Penrose
pseudo-inverse $\bm{A}^{+} \in \mathbb{R}^{m\times n}$ as the unique matrix
satisfying the following four properties \cite{penrose1955}:
\begin{subequations}
    \begin{align}
        &\textrm{1. } & \bm{A}\bm{A}^{+}\bm{A} &= \bm{A} \\
        &\textrm{2. } & \bm{A}^{+}\bm{A}\bm{A}^{+} &= \bm{A}^{+} \\
        &\textrm{3. } & (\bm{A}\bm{A}^{+})^T &= \bm{A}\bm{A}^{+} \\
        &\textrm{4. } & (\bm{A}^{+}\bm{A})^T &= \bm{A}^{+}\bm{A}
    \end{align}
\end{subequations}
Note that in the general case, the transpose operator can be replaced by the
conjugate transpose (hermitian) operator for matrices over the complex field $\mathbb{C}$ \cite{penrose1955}.
It can be shown that for a full-column rank matrix $\bm{A}$, the pseudo-inverse is
\begin{align}
    \bm{A}^{+} = (\bm{A}^T \bm{A})^{-1} \bm{A}^T,
\end{align}
and for a full-row rank matrix $\bm{A}$, the pseudo-inverse is
\begin{align}
    \bm{A}^{+} = \bm{A}^T (\bm{A} \bm{A}^T)^{-1}.
\end{align}
In the general case, whether or not $\bm{A}$ is full rank, the pseudo-inverse can always
be computed using the \gls{svd} of $\bm{A}$. The \gls{svd} of $\bm{A}$ is
\begin{subequations}
\begin{align}
    \bm{A} &= \bm{U} \bm{\Sigma} \bm{V}^T \\
    \bm{U} &\in \mathbb{R}^{n\times n} \textrm{ unitary} \\
    \bm{V} &\in \mathbb{R}^{m\times m} \textrm{ unitary} \\
    \bm{\Sigma} &= \operatorname{diag}(\sigma_1, \sigma_2, \ldots, \sigma_r, 0, \ldots, 0) \in \mathbb{R}^{n\times m} \\
    \sigma_1 &\geq \sigma_2 \geq \ldots \geq \sigma_r > 0 \\
    r &= \operatorname{rank}(\bm{A}).
\end{align}
\end{subequations}
Where a square matrix $U$ is unitary if and only if $U^T U = U U^T = \mathbb{I}$. The pseudo-inverse
of $\bm{A}$ can then be computed as
\begin{subequations}
\begin{align}
    \bm{A}^{+} &= \bm{V} \bm{\Sigma}^{+} \bm{U}^T \\
    \bm{\Sigma}^{+} &:= \operatorname{diag}(\sigma_1^{-1}, \sigma_2^{-1}, \ldots, \sigma_r^{-1}, 0, \ldots, 0).
\end{align}
\end{subequations}
The singular value decomposition will be important when dealing with
singularity-robust task-priority control later on in this thesis.

Consider the optimization problem
\begin{align}
    \min_{\bm{x} \in \mathbb{R}^m} || \bm{A} \bm{x} - \bm{b} ||.
\end{align}
The solution to this problem is given by
\begin{align}
    \bm{x} = \bm{A}^{+} \bm{b} + (\mathbb{I} - \bm{A}^{+} \bm{A}) \bm{z},
\end{align}
where $\bm{z} \in \mathbb{R}^n$ is an arbitrary vector. The term $(\mathbb{I} - \bm{A}^{+} \bm{A}) \bm{z}$
is the null-space projection of $\bm{z}$ onto the null-space of $\bm{A}$. Parts
of this fact can bee seen by noting that
\begin{align}
    \bm{A}\left(\mathbb{I} - \bm{A}^{+} \bm{A}\right) = \bm{A} - \bm{A} = 0.
\end{align}
This is a fact that will be used extensively when formulating the task-priority
controllers in later chapters.

% -----------------------------------------------------------------------------
\section{Lagrange's Equations of Motion}
\label{sec:lagrange}
Lagrange's equations of motion is a formulation of the equations of motion of a
system in terms of generalized coordinates. The equations are derived from the
d'Alembert principle. It can be shown that the equations of motion derived from
Lagrange's equations are equivalent to the equations of motion derived from
the classical Newton-Euler equations \cite{modsim}. There are however some advantages to using
Lagrange's equations. For example, the equations are easier to derive for complex
systems where the kinetic and potential energy is known. With modern computer
algebra systems, the equations can be derived automatically, which is
a great advantage for complex systems.

Consider a system with $n$ generalized coordinates $q_1, q_2, \ldots, q_n$, and
their time derivatives $\dot{q}_1, \dot{q}_2, \ldots, \dot{q}_n$. Collect the
generalized coordinates in the vector $\bm{q} \in \mathbb{R}^n$ and the time
derivatives in the vector $\bm{\dot{q}} \in \mathbb{R}^n$. The kinetic energy
of the system is then a function $T$ of $\bm{q}$ and $\bm{\dot{q}}$
\begin{align}
    T = T(\bm{q}, \bm{\dot{q}})
\end{align}
and the potential energy is a function $U$ of $\bm{q}$
\begin{align}
    U = U(\bm{q}).
\end{align}
Define the lagrangian $L$ as the difference between the kinetic and potential
energy
\begin{align}
    L(\bm{q}, \bm{\dot{q}}) = T(\bm{q}, \bm{\dot{q}}) - U(\bm{q}).
\end{align}
The equations of motion are then given by \cite{modsim}:
\begin{align}
    \frac{d}{dt} \left( \nabla_{\dot{\bm{q}}} L \right) - \nabla_{\bm{q}} L = \bm{\tau},
\end{align}
where the $i$-th element of the vector $\bm{\tau}$ is the generalized force acting
on the $i$-th generalized coordinate. The generalized forces can be expressed as
\begin{align}
    \tau_i &= \sum_{k=1}^N \frac{\partial \bm{r}_k}{\partial q_i} \cdot \bm{F}_k
    & i &= 1, 2, \ldots,
\end{align}
where $\bm{r}_k$ is the position of the $k$-th force and $\bm{F}_k$ is the force
acting on this position. In cases where the kinetic energy is on the form
\begin{align}
    T = \frac{1}{2} \bm{\dot{q}}^T \bm{J}^T(\bm{q}) \bm{M} \bm{J}(\bm{q}) \bm{\dot{q}},
\end{align}
where $\bm{M}$ is symmetric and positive definite, the potential energy the
result of some potential field, and $\bm{J}(\bm{q})$ is the Jacobian of the
generalized coordinates, to be defined later, the equations of motion can be
written as
\begin{align}
    \bm{M}(\bm{q}) \ddot{\bm{q}} + \bm{C}(\bm{q}, \dot{\bm{q}}) \dot{\bm{q}} + \bm{g}(\bm{q}) = \bm{\tau}
\end{align}
where $\bm{M}(\bm{q})$ is symmetric and positive definite, $\bm{g}(\bm{q})$ is the
gradient of the potential field, and $\bm{C}(\bm{q}, \dot{\bm{q}})$ models the
coriolis and centrifugal forces, which will be defined later.

\section{Runge-Kutta Methods}
\label{sec:runge-kutta}

Runge-Kutta methods are a well-known set of numerical techniques used to solve
ordinary differential equations. They are popular because they are
straightforward to use and usually quite efficient. In this thesis, these
methods will be applied to model the dynamics of underwater vehicle-manipulator
systems, which can be described by ordinary differential equations.
When simulating controllers for these systems, it makes sense to use Runge
-Kutta methods, since they provide a convenient way to handle the required
calculations.

Consider a system of ordinary differential equations of the form
\begin{align}
    \dot{\bm{x}}(t) &= \bm{f}(\bm{x}(t), t) & \bm{x}(t_0) &= \bm{x}_0,
\end{align}
where $\bm{x}(t) : \R \to \R^n$ is the state of the system,
$\bm{f} : \R^n \times \R \to \R^n$ is a vector field, $t_0$ is the initial time,
and $\bm{x}_0$ is the initial state. In general, the solution to this system is
not known analytically, and numerical methods, such as Runge-Kutta must be used
to approximate the solution. These methods start by initializing the state to
$\bm{x}_0$ and then iteratively update the state using the vector field $\bm{f}$,
in time steps of size $h$. For explicit Runge-Kutta methods, the approximation
is given by \cite{modsim}:
\begin{subequations}
    \begin{align}
        \bm{k}_i &= \bm{f}(\bm{x}_n + h \sum_{j=1}^{i-1} a_{ij}\bm{k}_j, t_n + c_i h) &
            i &= 1, 2, \cdots, \sigma \\
        \bm{x}_{n+1} &= \bm{x}_n + h \sum_{i=1}^{\sigma} b_i \bm{k}_i,
    \end{align}
\end{subequations}
where $\{a_{ij}\}_{i,j}$, $\{b_i\}_i$, and $\{c_i\}_i$ are the coefficients of
the method, and $\sigma \in \mathbb{N}$ is the number of stages. The error $\bm{e}_{n+1}$ of
the method at time $t_{n+1}$ is given by
\begin{align}
    \bm{e}_{n+1} = \bm{x}_{n+1} - \bm{x}(t_n;t_{n+1}),
\end{align}
where $\bm{x}(t_n;t_{n+1})$ is the exact solution at time $t_{n+1}$ given the
initial condition $\bm{x}_n$ at time $t_n$. We say that the method is of order
$p$, if $p \in \mathbb{N}$ is the smallest integer such that
\begin{align}
    ||\bm{e}_{n+1}|| = \mathcal{O}(h^{p+1}),
\end{align}
where $\mathcal{O}$ is the big-O notation \cite{modsim}.
The most well-known Runge-Kutta method is the fourth-order Runge-Kutta method,
which has the following coefficients:

\begin{align}
    \begin{matrix}
        c_1 & \vline & & & & \\
        c_2 & \vline & a_{21} & & & \\
        c_3 & \vline & a_{31} & a_{32} & & \\
        c_4 & \vline & a_{41} & a_{42} & a_{43} & \\
        \hline
        & \vline & b_1 & b_2 & b_3 & b_4
    \end{matrix}
    \quad=\quad 
    \begin{matrix}
        0 & \vline & & & & \\
        \frac{1}{2} & \vline & \frac{1}{2} & & & \\
        \frac{1}{2} & \vline & 0 & \frac{1}{2} & & \\
        1 & \vline & 0 & 0 & 1 & \\
        \hline
        & \vline & \frac{1}{6} & \frac{1}{3} & \frac{1}{3} & \frac{1}{6}
    \end{matrix}
\end{align}

This method is often referred to as RK4, and it is widely used due to its
balance between accuracy and computational efficiency. The method is
useful for simulating the dynamics of \gls{uvms}s.


\section{Quaternions}
\label{sec:background:quaternions}

There are several ways to parameterize rotations in three dimensions. One of 
the more intuitive approaches is through Euler angles. However, a key drawback 
of Euler angles is the occurrence of singularities when the pitch angle 
reaches $\pm90^\circ$, a phenomenon known as gimbal lock. While this may be 
acceptable in some applications, it poses challenges for general \gls{aiauv} 
systems. Another advantage of using quaternions is that they are often computationally more efficient and numerically stable. 

A widely used and effective alternative is to represent rotations using unit 
quaternions. Following Hamilton’s formulation, a quaternion is expressed as
\begin{align}
    \bm{q} = \eta + \varepsilon_1 \bm{i} + \varepsilon_2 \bm{j} + \varepsilon_3 \bm{k},
\end{align}
where $\eta \in \mathbb{R}$ is the scalar part, and \(\bm{i}, \bm{j}, \bm{k}\) 
form the vector (imaginary) part. The multiplication rules for these imaginary 
units are defined as
\begin{align}
    && \bm{i}^2 = \bm{j}^2 = &\bm{k}^2 = \bm{i}\bm{j}\bm{k} = -1 &&\\
    \bm{i}\bm{j} = &-\bm{j}\bm{i} = \bm{k} & \bm{j}\bm{k} = -&\bm{k}\bm{j} = \bm{i} & \bm{k}\bm{i} = -&\bm{i}\bm{k} = \bm{j}
\end{align}

For convenience, quaternions are often written in vector form:
\begin{align}
    \bm{q} &= \begin{bmatrix}
        \eta \\
        \bm{\varepsilon}
    \end{bmatrix} & \bm{\varepsilon} &= \begin{bmatrix}
        \varepsilon_1 \\
        \varepsilon_2 \\
        \varepsilon_3
    \end{bmatrix}
\end{align}

The quaternion product \(\otimes\) is defined as follows:
\begin{align}
    \bm{q}_1 \otimes \bm{q}_2 &=
    \begin{bmatrix}
        \eta_1 \\
        \bm{\varepsilon}_1
    \end{bmatrix} \otimes
    \begin{bmatrix}
        \eta_2 \\
        \bm{\varepsilon}_2
    \end{bmatrix} = 
    \begin{bmatrix}
        \eta_1\eta_2 - \bm{\varepsilon}_1^T\bm{\varepsilon}_2 \\
        \eta_1\bm{\varepsilon}_2 + \eta_2\bm{\varepsilon}_1 + \bm{\varepsilon}_1\times\bm{\varepsilon}_2
    \end{bmatrix} 
\end{align}

A quaternion is considered a unit quaternion if its norm equals one:
\begin{align}
    ||\bm{q}||_2 = \sqrt{\eta^2 + \bm{\varepsilon}^T \bm{\varepsilon}} = 1.
\end{align}

Unit quaternions are particularly useful because they can represent any 
rotation in three-dimensional space. This can be shown by expressing a 
quaternion in terms of a rotation angle \(\theta\) and an axis \(\bm{k}\), 
where \(\bm{k}\) is a unit vector:
\begin{align}
    \begin{bmatrix}
        \eta \\
        \bm{\varepsilon}
    \end{bmatrix} = 
    \begin{bmatrix}
        \cos\left(\frac{\theta}{2}\right) \\
        \sin\left(\frac{\theta}{2}\right)\bm{k}
    \end{bmatrix}. \label{eq:quaternion_angle_axis}
\end{align}

By applying \autoref{eq:quaternion_angle_axis} and Rodrigues’ formula, the 
rotation matrix \(\bm{R}\) corresponding to a quaternion \(\bm{q}\) can be 
written as:
\begin{align}
    \bm{R} &= \I + \sin\theta [\bm{k}]_{\times} + (1-\cos\theta)[\bm{k}]_{\times}[\bm{k}]_{\times} \\
    &= \I + 2 \eta [\bm{\varepsilon}]_{\times} + 2 [\bm{\varepsilon}]_{\times} [\bm{\varepsilon}]_{\times}.
\end{align}

When using unit quaternions, the quaternion product represents the composition 
of the two corresponding rotations. As a result, the product is associative 
but not commutative. The inverse of a unit quaternion, which corresponds to the 
inverse rotation, is simply its conjugate:
\begin{align}
    \bm{q}^* = \begin{bmatrix}
        \eta \\
        \bm{\varepsilon}
    \end{bmatrix}^* = \begin{bmatrix}
        \eta \\
        -\bm{\varepsilon}
    \end{bmatrix}.
\end{align}
This is clear from looking at the angle-axis comparison shown in \autoref{eq:quaternion_angle_axis}.
These equations highlight the computational advantages of using quaternions 
for rotation representation. Operations such as inversion and multiplication 
reduce to simple calculations in \(\R^4\). Additionally, unlike other 
parameterizations, quaternions avoid kinematic singularities.

The relationship between the time derivative of the quaternion, \(\dot{\bm{q
}}\), and the body-fixed angular velocity \(\bm{\omega}^b\) is given by
\cite{modsim}:
\begin{align}
    \dot{\eta} &= -\frac{1}{2} \bm{\varepsilon}^T \bm{\omega}^b &
    \dot{\bm{\varepsilon}} &= \frac{1}{2} \left(\eta \I + [\bm{\varepsilon}]_{\times}\right),
\end{align}
which can be written in matrix form as:
\begin{align}
    \dot{\bm{q}} = \frac{1}{2} \begin{bmatrix}
        -\bm{\varepsilon}^T \\
        \I + [\bm{\varepsilon}]_{\times}
    \end{bmatrix}
    \bm{\omega}^b
\end{align}
Notably, the bottom \(3 \times 3\) block of this matrix has an inverse:
\begin{align}
    \left(
        \I + [\bm{\varepsilon}]_{\times}
    \right)^{-1}
    = 
    \I + \frac{1}{1+||\bm{\varepsilon}||_2^2}\left([\bm{\varepsilon}]_{\times}[\bm{\varepsilon}]_{\times}
    - [\bm{\varepsilon}]_{\times}\right),
\end{align}
which confirms that the kinematic equation is well-posed and free from 
singularities. Concepts from this section will be foundational when 
formulating the kinematic equations for \gls{aiauv}s. They will also be 
important in the development of the lower-level \gls{pd} controller in
\autoref{sec:tpc:low_level_control}, and when defining task Jacobians 
for experiments.


