\chapter{Conclusion \& Future Work}
\label{ch:conclusion}

% ------------------------------------------------------------------ Conclusion
\section{Conclusion}

\begin{itemize}
    \item \textbf{Controll, visualization and simulation software}:
    \item \textbf{Dynamic-positioning controller}
    \item \textbf{Velocity Level \gls{tpc}}:
    \item \textbf{Set-Based \gls{tpc}}:
\end{itemize}

% ----------------------------------------------------------------- Future Work
\section{Future Work}
\label{sec:conclusion:future_work}

While this thesis demonstrates the feasibility of kinematic-level \gls{tpc} on a lightweight, highly redundant underwater robot, several areas remain open for further exploration and development. The work presented here lays the foundation for multiple extensions and improvements, both in terms of modeling accuracy and control strategy sophistication. Potential directions for future work include:

\begin{itemize}
    \item \textbf{Model identification}: Although an approximate dynamic model was developed as part of this thesis, further improvements are needed to enhance its fidelity. In particular, accurate estimation of added mass and hydrodynamic damping remains challenging. Dedicated experimental procedures could be designed to more precisely identify these parameters, significantly improving model-based control strategies.

    \item \textbf{Dynamic-level \gls{tpc}}: A natural extension of the work presented here is the implementation of \gls{tpc} at the dynamic level. This would require a significantly more accurate dynamic model, as control inputs would be computed based on full system dynamics rather than velocity kinematics alone. With a sufficiently precise model, dynamic-level TPC may offer improved performance in terms of responsiveness and disturbance rejection.

    \item \textbf{Controller tuning}: There is considerable scope for improving performance through systematic tuning of both the low-level \gls{pd+} controller and the high-level \gls{tpc} parameters. In particular, the proportional gains for individual tasks could be better adapted to system behavior. Additionally, since the robot's local dynamics vary with configuration, implementing a gain-scheduled or adaptive control strategy could further enhance tracking performance.

    \item \textbf{Rejection of joint measurement jitter}: Joint angle jitter was partially mitigated in this thesis using low-pass filtering. However, more advanced techniques—such as outlier detection, sensor fusion, or model-based filtering—could allow for more effective rejection of faulty measurements without introducing significant time delays.

    \item \textbf{Advanced control strategies}: The software and experimental framework developed in this thesis enables further research into advanced control techniques. Methods such as adaptive control, learning-based strategies, or alternative task-priority formulations could be implemented and tested using the same platform. This opens up new avenues for exploration and motivates continued development of the system.
\end{itemize}

