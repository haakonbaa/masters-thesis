\chapter{Results \& Discussion}

The following chapter presents the experimental results alongside a discussion 
of the findings. It begins with an overview of the experimental runs, followed 
by selected outcomes from each. First, the performance of the \gls{pd+} controller 
is examined, with step response results shown for heading, roll, and both 
north and east position control. Next, the results from experiments using the 
task-priority controller are presented, focusing initially on the scenario in which the 
robot is stretched. This is followed by results from experiments where the 
robot is bent, utilizing a set-based task-priority controller. Finally, a comparison is 
made between runs with and without low-pass filtering of the generalized velocities.


\section{Overview of Results}

Each experimental run is classified by the time it was started, and there are a total
of \(10\) runs that will be presented in the following sections. The format
of the run name is \texttt{YYYYMMDD-HHMMSS}, where the first part is the date
and the second part is the time. An overview of the runs is presented in
\autoref{tab:eelume:experimental-runs}, which includes the run name, the
controller used, if the velocity measurements were low-pass filtered, and a
brief description of each run.

\begin{table}[!ht]
    \centering
    \begin{tabular}{|c|c|c|c|}
        \hline
        Run Name & Controller & Low-pass & Case \\ \hline \hline
        \multirow{2}{*}{20250513-135042} & \gls{pd+} & No & Heading \\ \cline{2-4}
        & \multicolumn{3}{p{0.75\linewidth}|}{Reference in yaw angle set to \(20\) degrees after 20 seconds.} \\ \hline
        \multirow{2}{*}{20250513-141406} & \gls{pd+} & No & Roll \\ \cline{2-4}
        & \multicolumn{3}{p{0.75\linewidth}|}{Reference in roll angle set to \(20\) degrees after 20 seconds.} \\ \hline
        \multirow{2}{*}{20250513-133631} & \gls{pd+} & No & North Position \\ \cline{2-4}
        & \multicolumn{3}{p{0.75\linewidth}|}{Reference in north position set to \(+1\) m after 20 seconds.} \\ \hline
        \multirow{2}{*}{20250513-134500} & \gls{pd+} & No & East Position \\ \cline{2-4}
        & \multicolumn{3}{p{0.75\linewidth}|}{Reference in east position set to \(+1\) m after 20 seconds.} \\ \hline
        \hline
        \multirow{2}{*}{20250519-141527} & Kinematic-level \gls{tpc} & Yes & Stretch \\ \cline{2-4}
        & \multicolumn{3}{p{0.75\linewidth}|}{Showing case \textit{stretch} with decoupled quaternion gains.} \\ \hline
        \multirow{2}{*}{20250519-144113} & Kinematic-level \gls{tpc} & No & Stretch \\ \cline{2-4}
        & \multicolumn{3}{p{0.75\linewidth}|}{Showing case \textit{stretch} with no compensation term.} \\ \hline
        \hline
        \multirow{2}{*}{20250519-161629} & Kinematic-Level \gls{tpc} & Yes & Bend \\ \cline{2-4}
        & \multicolumn{3}{p{0.75\linewidth}|}{Showing case \textit{bend}, with compensation term.} \\ \hline
        \multirow{2}{*}{20250519-161107} & Kinematic-Level \gls{tpc} & No & Bend \\ \cline{2-4}
        & \multicolumn{3}{p{0.75\linewidth}|}{Showing case \textit{bend}, without compensation term.} \\ \hline
        \hline
        \multirow{2}{*}{20250513-154920} & Set-based \gls{tpc} & No & Stretch \\ \cline{2-4}
        & \multicolumn{3}{p{0.75\linewidth}|}{Showing case \textit{stretch} with no low-pass filtering.} \\ \hline
        \multirow{2}{*}{20250519-135739} & Set-based \gls{tpc} & Yes & Stretch \\ \cline{2-4}
        & \multicolumn{3}{p{0.75\linewidth}|}{Showing case \textit{stretch} with low-pass filtering.} \\ \hline
    \end{tabular}
    \caption[An overview of the experimental runs]{An overview of the experimental
    runs presented in this chapter along with a brief description of each.
    The low-pass column indicates whether or not the generalized velocities and joint measurements were low-pass filtered.}
    \label{tab:eelume:experimental-runs}
\end{table}

% -----------------------------------------------------------------------------
\section{DP-Control}

The following sections show the results from testing the \gls{pd+} controller
on the Eelume robot. For a detailed overview of the setup, see \autoref{sec:experimental_setup:experimental_procedures}.

\subsection{Heading Control}

\begin{figure}[!ht]
    \centering
    \includegraphics[width=0.8\textwidth,page=1]{assets/ignored/plots/pd.pdf}
    \caption{Position of Eelume robot during heading control with \gls{pd+} controller. Run \textit{20250513-135042}}
    \label{fig:results:dp_heading:pos}
\end{figure}
\begin{figure}[!ht]
    \centering
    \includegraphics[width=0.8\textwidth,page=2]{assets/ignored/plots/20250513-135042.pdf}
    \caption{Generalized velocity of Eelume robot during heading control with \gls{pd+} controller. Run \textit{20250513-135042}}
    \label{fig:results:dp_heading:vel}
\end{figure}
\begin{figure}[!ht]
    \centering
    \includegraphics[width=0.8\textwidth,page=5]{assets/ignored/plots/20250513-135042.pdf}
    \caption{Thruster forces and joint motor torques of Eelume robot during heading control with \gls{pd+} controller. Run \textit{20250513-135042}}
    \label{fig:results:dp_heading:forces2}
\end{figure}

\paragraph{Results.}
\autoref{fig:results:dp_heading:pos} shows the position and orientation of the 
Eelume robot during a heading control task. The robot is commanded to turn to 
a heading of \(20^\circ\) at \(t = 20\) seconds. The yaw angle gradually 
converges to the reference, with an overshoot of approximately \(3^\circ\), 
followed by damped oscillations. The system settles within a few degrees of 
the reference after approximately \(25\) seconds. The roll and pitch angles 
remain near \(0^\circ\), indicating stable attitude in those directions.

The \gls{ned} position of the robot, which is controlled to remain near the 
origin, is somewhat affected by the heading maneuver. Both the north and down 
positions exhibit a steady-state offset of approximately \(0.3\,\mathrm{m}\) 
from the reference. This suggests a degree of coupling between the heading and 
translational dynamics. The joint angle measurements also show considerable 
high-frequency noise.

The generalized velocities, shown in \autoref{fig:results:dp_heading:vel}, 
further highlight this noise. While the angular velocities are visibly noisy, 
the joint velocities are particularly affected, with high-frequency 
oscillations and spikes reaching up to \(100^\circ/\mathrm{s}\).

\autoref{fig:results:dp_heading:forces2} presents the resulting thruster 
forces and joint motor torques. Although the thruster forces contain some high-
frequency noise, its amplitude is moderate compared to that of the generalized 
velocities. The joint torques, however, exhibit significant noise, frequently 
reaching the actuator limit of \(\pm16\,\mathrm{Nm}\).

\paragraph{Discussion.}

The results in \autoref{fig:results:dp_heading:pos} confirm that the \gls{pd+} 
controller is capable of driving the heading of the Eelume robot to a desired 
value. However, the response is relatively slow, which can be attributed to 
the substantial hydrodynamic forces resisting rotation about the body \(z\)-
axis. The observed overshoot and oscillatory behavior suggest that further 
tuning of the controller gains could improve performance. The robot's large 
inertia around the \(z\)-axis likely contributes to the sluggish response.

An important observation is the degradation in position-holding performance 
during the heading change. This likely results from coupling between the 
degrees of freedom, as the thrusters must simultaneously handle both position 
and heading control.

The joint angle noise, noted in \autoref{fig:results:dp_heading:pos}, aligns 
with the joint velocity spikes observed in \autoref{fig:results:dp_heading:vel
}. These artifacts are consistent with the "jittering" behavior described in
\autoref{sec:experimental_setup:assumptions_and_limitations}. The presence of 
similar high-frequency noise in the thruster forces and joint torques
(\autoref{fig:results:dp_heading:forces2}) can be explained by the unfiltered 
propagation of noisy joint and velocity measurements through the controller 
and thruster allocation algorithms.

Of particular concern is the behavior of the joint motor torques, which 
resemble a "bang-bang" control strategy rather than a smooth \gls{dp} response
. This is partly due to the high proportional gain in the controller—set to
\(10\,\mathrm{Nm}/^\circ\)—meaning that a \(1^\circ\) error induces a 
\(10\,\mathrm{Nm}\) corrective torque. While this gain yields low joint angle 
errors, it also results in aggressive, high-frequency control actions that may 
not be ideal for the mechanical components and actuators. Future work should 
consider the use of low-pass filtering or gain scheduling to mitigate these effects.

\FloatBarrier

% -----------------------------------------------------------------------------
\subsection{Roll Control}

\begin{figure}[!ht]
    \centering
    \includegraphics[width=0.8\textwidth,page=2]{assets/ignored/plots/pd.pdf}
    \caption{Position of Eelume robot during roll control with \gls{pd+} controller.}
    \label{fig:results:dp_roll:pos}
\end{figure}

\paragraph{Results.}

\autoref{fig:results:dp_roll:pos} shows the orientation and position of the 
Eelume robot during a roll control test. The robot is commanded to achieve a 
roll angle of approximately \(20^\circ\), which it tracks with relatively high 
accuracy. The response resembles that of a second-order system, with a minor 
overshoot of about \(2^\circ\) and settling occurring within approximately
\(3\) seconds.

Throughout the maneuver, pitch and yaw remain close to the reference value of
\(0^\circ\), deviating by no more than \(2^\circ\), indicating good decoupling 
of the rotational axes. The north and east position components show small 
oscillations around the reference point, with deviations of no more than
\(0.2\,\mathrm{m}\). However, the down (vertical) position does not converge 
to the origin but instead stabilizes at a depth between \(0.4\) and
\(0.5\,\mathrm{m}\) below the intended reference.

As in the previous experiment, the joint angle measurements exhibit noticeable 
noise and jitter, which may affect control quality in high-frequency domains.

\paragraph{Discussion.}

The roll angle tracking performance is notably precise and significantly 
faster than the yaw response observed in \autoref{fig:results:dp_heading:pos}. 
This improved performance is primarily due to the robot’s lower moment of 
inertia about the body-fixed \(x\)-axis. Additionally, the Eelume’s slender, 
snake-like design minimizes hydrodynamic drag about this axis, except for 
minor contributions from external components such as thrusters and instrumentation.

The slight steady-state error in roll angle may be attributed to the robot’s 
buoyancy characteristics. As the Eelume has a stable roll equilibrium due to 
its design, restoring moments from buoyancy counteract the control effort, 
resulting in a residual error when the roll angle is held away from neutral.

The observed depth error likely stems from inaccuracies in the compensation 
term within the controller. Furthermore, the test was conducted at a depth of 
approximately \(1.3\,\mathrm{m}\), where variations in salinity and 
temperature can influence buoyancy. Performing the experiment at a greater 
depth could have reduced the impact of surface-related density fluctuations.

The noise and jitter in joint measurements are consistent with those observed 
in the heading control experiment. While they do not appear to significantly 
degrade roll tracking performance, they remain a potential concern for control 
fidelity and actuator health, particularly in dynamic scenarios.

Overall, the experiment demonstrates effective and responsive roll control, 
validating the controller’s ability to handle fast and accurate orientation 
adjustments about the \(x\)-axis.

\FloatBarrier


% -----------------------------------------------------------------------------
\subsection{North Position Control}

\begin{figure}[!ht]
    \centering
    \includegraphics[width=0.8\textwidth,page=3]{assets/ignored/plots/pd.pdf}
    \caption{Position of Eelume robot during north position control with \gls{pd+} controller.}
    \label{fig:results:dp_north:pos}
\end{figure}

\begin{figure}[!ht]
    \centering
    \includegraphics[width=0.8\textwidth,page=5]{assets/ignored/plots/20250513-133631.pdf}
    \caption{Thruster forces and joint motor torques during north position control with \gls{pd+} controller.}
    \label{fig:results:dp_north:forces}
\end{figure}

\paragraph{Results.}

\autoref{fig:results:dp_north:pos} illustrates the position and orientation of the Eelume robot in response to a step change of \(1\,\mathrm{m}\) in the north reference. The northward motion exhibits a second-order response, with an overshoot of approximately \(0.4\,\mathrm{m}\), and settles near the desired value within about \(40\) seconds. The response is notably slow, which may limit the system's ability to follow time-sensitive trajectories.

Throughout the experiment, roll, pitch, and yaw remain within a stable range of \(\pm 2^\circ\) after the initial transient phase, indicating good attitude stability and decoupling from the translational motion.

Anomalies were observed in the joint angle measurements, particularly in joint \(\theta_3\). Two distinct jittering episodes occur between \(t = 18\text{–}26\,\mathrm{s}\) and \(t = 41\text{–}47\,\mathrm{s}\), during which the third joint appears to introduce random jumps of approximately \(4^\circ\). These irregularities are reflected in the control signals.

\autoref{fig:results:dp_north:forces} shows the corresponding actuator commands. As in previous experiments, the thruster forces and joint torques exhibit high-frequency noise. The noise is particularly pronounced during the intervals of joint jitter, further supporting the observation of measurement-induced disturbances.

\paragraph{Discussion.}

The slow rise time and overshoot in the northward response suggest a lack of damping and conservative controller tuning in the translational plane. A higher proportional gain could reduce rise time, while increased derivative action may help suppress overshoot.

The jitter observed in the joint measurements—especially in joint \(\theta_3\)—is not unique to this experiment and appears to be a recurring issue. Although the exact cause remains unclear, it is discussed in more detail in \autoref{sec:results:lowpass}, where low-pass filtering is proposed as a mitigation strategy. It is evident that these joint disturbances contribute to the high-frequency noise seen in both the joint motor commands and thruster allocation.

Despite the noise, the joint angles perform well overall, provided the brief jittering segments are excluded. As in earlier experiments, a steady-state error in the down position persists, likely due to buoyancy-related inaccuracies and the limited effectiveness of the vertical compensation term in the controller. This is further affected by variations in water salinity near the surface.

In summary, while the system demonstrates successful convergence to the north position reference, performance is hindered by slow dynamics and sensor-related disturbances. These findings underscore the importance of controller tuning and measurement filtering to ensure reliable and responsive behavior.

\FloatBarrier

% -----------------------------------------------------------------------------
\subsection{East Position Control}
\begin{figure}[!ht]
    \centering
    \includegraphics[width=0.8\textwidth,page=4]{assets/ignored/plots/pd.pdf}
    \caption{Position of Eelume robot during heading control with \gls{pd+} controller.}
    \label{fig:results:dp_east:pos}
\end{figure}

\paragraph{Results.}

\autoref{fig:results:dp_east:pos} shows the response of the Eelume robot to a 
\(1\,\mathrm{m}\) step in the east position reference. The eastward motion 
exhibits characteristics of an over-damped second-order system. The robot 
reaches approximately \(0.9\,\mathrm{m}\) after about \(20\) seconds, with a 
residual steady-state error of approximately \(0.1\,\mathrm{m}\), indicating 
that the controller does not fully eliminate the position error in this 
direction.

Small steady-state offsets are also observed in the north and down directions, 
with final errors of approximately \(0.2\,\mathrm{m}\) and \(0.3\,\mathrm{m}\),
respectively. These secondary deviations may result from coupling effects 
or imbalances in controller compensation.

The roll and yaw angles show brief disturbances during the initial phase of 
the motion but stabilize within the first \(10\) seconds. This transient 
effect indicates that the system momentarily perturbs the robot’s orientation 
when initiating lateral movement, before regaining stable attitude control.

The joint angles remain stable throughout the run, apart from the usual high-
frequency jitter observed in other experiments. No major anomalies or 
uncharacteristic behavior were identified.

\paragraph{Discussion.}

The sluggish response and steady-state error in the east direction are 
expected given the Eelume robot's physical configuration. Translating 
laterally requires the entire length of the slender, snake-like body to 
displace water sideways, resulting in significant hydrodynamic drag. This 
introduces large damping forces that slow the robot's response in the east 
direction compared to forward (north) motion.

The observed behavior aligns well with theoretical expectations. The over-
damped response profile and incomplete convergence are indicative of both 
underpowered control effort and substantial resistance from the surrounding 
fluid. While the robot is able to execute the motion, the performance may be 
limited by actuation constraints or the need for refined controller tuning to 
handle such high-drag scenarios more effectively.

Overall, the results demonstrate that while eastward translation is 
challenging for the Eelume platform, the control system maintains stability 
and achieves acceptable performance, albeit with slower dynamics and modest 
steady-state error.

\FloatBarrier

% -----------------------------------------------------------------------------
\newpage
\section{Velocity-Level Task-Priority Control}

\begin{figure}[!ht]
    \centering
    \includegraphics[width=0.8\textwidth,page=3]{assets/ignored/plots/20250519-141527.pdf}
    \caption[Low-pass filtered position of Eelume robot during \textit{stretch} case with \gls{tpc}]{Low-pass filtered position of Eelume robot during \textit{stretch} case with \gls{tpc}. The position is offset from \((11, 36, 10)\)}
    \label{fig:results:tpc:stretch:1:pos}
\end{figure}
\begin{figure}[!ht]
    \centering
    \includegraphics[width=0.8\textwidth,page=4]{assets/ignored/plots/20250519-141527.pdf}
    \caption{Low-pass filtered generalized velocity of Eelume robot during \textit{stretch} case with \gls{tpc}.}
    \label{fig:results:tpc:stretch:1:vel}
\end{figure}
\begin{figure}[!ht]
    \centering
    \includegraphics[width=0.8\textwidth,page=6]{assets/ignored/plots/20250519-141527.pdf}
    \caption{}
    \label{fig:results:tpc:stretch:1:forces}
\end{figure}
\begin{figure}[!ht]
    \centering
    \includegraphics[width=0.8\textwidth,page=5]{assets/ignored/plots/20250519-141527.pdf}
    \caption{}
    \label{fig:results:tpc:stretch:1:forces-torques}
\end{figure}
\begin{figure}[!ht]
    \centering
    \includegraphics[width=0.8\textwidth,page=7]{assets/ignored/plots/20250519-141527.pdf}
    \caption{}
    \label{fig:results:tpc:stretch:1:task:1}
\end{figure}
\begin{figure}[!ht]
    \centering
    \includegraphics[width=0.8\textwidth,page=8]{assets/ignored/plots/20250519-141527.pdf}
    \caption{}
    \label{fig:results:tpc:stretch:1:task:2}
\end{figure}
\begin{figure}[!ht]
    \centering
    \includegraphics[width=0.8\textwidth,page=9]{assets/ignored/plots/20250519-141527.pdf}
    \caption{}
    \label{fig:results:tpc:stretch:1:task:3}
\end{figure}

\FloatBarrier

% -----------------------------------------------------------------------------
\section{Set-Based Task-Priority Control}





\subsection{Without Compensation Term}

\begin{figure}[!ht]
    \centering
    \includegraphics[width=0.8\textwidth,page=7]{assets/ignored/plots/20250519-161107.pdf}
    \caption{}
    \label{fig:results:tpc:bend:1:pos}
\end{figure}

\begin{figure}[!ht]
    \centering
    \includegraphics[width=0.8\textwidth,page=8]{assets/ignored/plots/20250519-161107.pdf}
    \caption{}
    \label{fig:results:tpc:bend:1:pos}
\end{figure}

\begin{figure}[!ht]
    \centering
    \includegraphics[width=0.8\textwidth,page=9]{assets/ignored/plots/20250519-161107.pdf}
    \caption{}
    \label{fig:results:tpc:bend:1:pos}
\end{figure}



\FloatBarrier


\subsection{With Compensation Term}

\begin{figure}[!ht]
    \centering
    \includegraphics[width=0.8\textwidth,page=7]{assets/ignored/plots/20250519-161629.pdf}
    \caption{}
    \label{fig:results:tpc:bend:1:pos}
\end{figure}

\begin{figure}[!ht]
    \centering
    \includegraphics[width=0.8\textwidth,page=8]{assets/ignored/plots/20250519-161629.pdf}
    \caption{}
    \label{fig:results:tpc:bend:1:pos}
\end{figure}

\begin{figure}[!ht]
    \centering
    \includegraphics[width=0.8\textwidth,page=9]{assets/ignored/plots/20250519-161629.pdf}
    \caption{}
    \label{fig:results:tpc:bend:1:pos}
\end{figure}

\FloatBarrier


% -----------------------------------------------------------------------------
\section{Low-pass Filtering}
\label{sec:results:lowpass}
\begin{figure}[h!]
    \centering
    \includegraphics[width=0.8\textwidth,page=9]{assets/ignored/plots/20250519-141527.pdf}
    \caption{}
    \label{fig:results:tpc:stretch:1:task:3}
\end{figure}


\label{sec:results:lowpass_filtering}
