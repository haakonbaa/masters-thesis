Undervanns farkost-manipulator systemer (UVMSs) brukes i økende grad til komplekse undervannsoppgaver som inspeksjon, kartlegging, vedlikehold og reparasjon. Slike operasjoner innebærer ofte at man må håndtere flere samtidige mål, for eksempel å kontrollere posisjonen og orienteringen til manipulator-armen, samtidig som man opprettholder gode synsvinkler fra kameraer og lys montert andre steder på UVMS-en.

I tillegg byr lette UVMS-er på særegne kontrollutfordringer på grunn av sterk kobling mellom basen og manipulatorarmen, noe som gjør at tradisjonelle, separerte kontrollmetoder blir mindre effektive.

En lovende tilnærming for å håndtere disse utfordringene er Task-Priority Control (TPC), en metode som tillater eksplisitt prioritering av kontrolloppgaver basert på deres relative viktighet. Denne oppgaven undersøker anvendelsen av TPC for UVMS-er, med særlig fokus på praktisk implementering og evaluering av TPC-rammeverk på kinematisk nivå.

Som et bidrag til eksperimentell validering av et TPC-rammeverk, utvikles et dedikert C++ kontrollbibliotek for Eelume-roboten—en sterkt koblet, slangelignende undervannsrobot med mange frihetsgrader. Biblioteket inkluderer en innebygd simulator og støtter både utvikling og utprøving av kontrollstrategier.

Eksperimenter gjennomført i Trondheimsfjorden som en del av arbeidet viser at TPC på kinematisk nivå fungerer effektivt når det anvendes på UVMS-er. Resultatene bekrefter ikke bare den praktiske anvendbarheten av metoden, men legger også et solid grunnlag for videre forskning på TPC på dynamisk nivå og utforskning av alternative avanserte kontrollstrategier med Eelume-plattformen.
