This thesis was created as part of my Master's at the Norwegian University of Science and Technology (NTNU) in Trondheim, Norway, during the spring of 2025. It completes the five-year Master's program in Cybernetics and Robotics at the Faculty of Information Technology and Electrical Engineering.

This thesis is a continuation of my unpublished project thesis, \say{Task-Priority Control of Underwater Vehicle-Manipulator Systems}, written in the autumn of 2024. As such, some parts of this thesis are based primarily on the work done in that project. The following list
summarizes the parts of this thesis that are based on the project thesis, as well as the changes made to those parts.

\begin{itemize}
    \item \autoref{ch:introduction}; \autoref{sec:introduction:motivation} and \autoref{sec:introduction:literature}.
    \item \autoref{ch:background}; All sections except for \autoref{sec:background:quaternions}, which was added to provide some context for the quaternion-based dynamic model and control laws.
    \item \autoref{chap:modeling}; All sections are based on the project thesis, but include changes related to incorporating quaternions into the model.
    \item \autoref{chap:tpc}; This chapter has been rewritten to better align with the standard representation of kinematics and dynamics of \glspl{uvms} as presented in the literature.
\end{itemize}

AI tools were employed during the preparation of this report. Specifically, AI was utilized for spell-checking, proofreading, and text generation. However, the concepts, methodologies, and approaches presented in this thesis are entirely my own or were developed in collaboration with my supervisors. Additionally, certain segments of code used to generate results were created with the assistance of ChatGPT and GitHub Copilot.

Finally, I would like to thank my co-supervisors, Markus H. Iversflaten and Bjørn K. Sæbø, for all their help and support during the project. Many days were spent together conducting experiments at Trondheim Biological Station, and I am very grateful for all the time they devoted to assisting me. I am also grateful to my friend and fellow student, Tor Rassmussen, for the many insightful conversations and support during the project.
