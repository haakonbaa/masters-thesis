\Glspl{uvms} are increasingly utilized for complex underwater tasks, including inspection, mapping, maintenance, and repair.
Typical operations require managing multiple simultaneous objectives, such as controlling the position and orientation of an end-effector while maintaining optimal viewpoints from cameras and lighting systems mounted elsewhere on the \gls{uvms}.
Additionally, light-\glspl{uvms} pose unique control challenges due to significant coupling between the base vehicle and its manipulator arm, rendering decoupled control approaches less effective.

A promising solution to address these challenges is \gls{tpc}, an approach that allows explicit prioritization of tasks according to their relative importance. This thesis explores the application of \gls{tpc} to \glspl{uvms}, focusing specifically on the practical implementation and evaluation of kinematic-level \gls{tpc} frameworks.

This thesis contributes to the experimental verification of a \gls{tpc} framework by developing a dedicated C++ control library for the Eelume robot—a highly coupled, high-\gls{dof}, snake-like robot. The library includes a built-in simulator and supports both development and deployment of control strategies.

Experiments conducted in the Trondheim fjord as part of the thesis demonstrate the efficacy of kinematic-level \gls{tpc} when applied to \glspl{uvms}.
These results not only validates the practical applicability of kinematic-level \gls{tpc} but also lays a robust foundation for future research into dynamic-level \gls{tpc} and the exploration of alternative advanced control strategies using the Eelume robot.

