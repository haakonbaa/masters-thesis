\label{ch:introduction}
This chapter introduces the core motivation behind the thesis and outlines the
context and importance of the work. The driving factors and key challenges that
led to the thesis are presented. A brief overview of the literature that is
relevant to the thesis is given, and the assumptions made are
stated. The chapter concludes with a summary of the contributions of this thesis,
as well as an outline of the subsequent chapters.

% ------------------------------------------------------------------ Motivation
\section{Motivation}
\label{sec:introduction:motivation}


Lightweight \glspl{uvms}, and specifically \glspl{aiauv}, are highly kinematically redundant systems, meaning they possess more controllable \glspl{dof} than are strictly required for a given task. While this redundancy offers flexibility in task execution—allowing the system to avoid obstacles, optimize energy consumption, and maintain stability—it also introduces significant control challenges. Traditional control methods often fall short when managing the coupling effects between the vehicle's base and its manipulator arms, especially in lightweight systems where coupling is more pronounced.

\Gls{tpc} emerges as an effective solution by enabling the simultaneous execution of multiple control objectives, ordered by priority. While TPC has been extensively studied in the literature, much of the existing research focuses on heavy, work-class \glspl{uvms} or robotic arms in isolation. The specific challenges associated with light-UVMSs remain relatively underexplored. These systems pose additional difficulties due to the complexity of accurately modeling hydrodynamic forces and the greater sensitivity of the system to disturbances, making them harder to control to a given level of precision.
This thesis contributes to the limited body of research on TPC for lightweight, highly redundant systems by conducting experimental validation of velocity-level TPC on the Eelume 500—a high-\gls{dof}, strongly coupled, lightweight \gls{aiauv}. Through this work, the practical applicability of TPC in challenging real-world underwater environments is examined.

The development of advanced control strategies also requires a reliable and realistic simulation environment to test and validate performance before deploying controllers on physical systems. For light-\gls{uvms}s, accurately capturing hydrodynamic effects, coupling behavior, and actuator dynamics is crucial to minimize the simulation-to-reality gap. Furthermore, the high computational demands of dynamic-level TPC algorithms necessitate efficient and responsive simulation tools.
To support this, a simulation and control environment tailored for the Eelume 500 robot was developed as part of this thesis. This framework enables seamless transition between simulation and real-world experiments and supports real-time execution of kinematic-level TPC strategies.

In 2015, the United Nations adopted the 2030 Agenda for Sustainable Development
\cite{UN2030Agenda}, which outlines 17 Sustainable Development Goals (SDGs) to
address global challenges such as poverty, inequality, climate change, and
environmental degradation. The research and methods presented in this thesis
contribute to the achievement of several of these goals. For example, \emph{SDG 14
Life Below Water}, "Conserve and sustainably use the oceans, seas and marine
resources for sustainable development.". \gls{uvms}s can monitor and inspect underwater
infrastructure, reducing the risk associated with oil spills, structural failures,
or leaks. Furthermore, \gls{uvms}s controlled in a task-priority manner can collect automated
high resolution imagery and data, identifying and mitigating potential
damage and disruptions to marine ecosystems.

\emph{SDG 9 Industry, Innovation, and Infrastructure} aims to "Build resilient
infrastructure, promote inclusive and sustainable industrialization and foster
innovation. This can be achieved by enhancing underwater infrastructure reliability
through regular and automated inspection and maintenance, reducing the risk of
structural failures and environmental damage. This thesis also promotes innovation
in the field of maritime robotics, leading to more advanced technologies for
sustainable ocean management.

Finally, \emph{SDG 13 Climate Action} aims to "Take urgent action to combat climate
change and its impacts". \gls{uvms}s can facilitate oceanographic research by collecting
environmental data and monitoring climate change, laying the groundwork for actions
to combat the effects of climate change. \gls{uvms}s can support the development of
sustainable marine energy solutions, such as offshore wind farms and wave energy.

In summary, by aligning with key SDGs, this research demonstrates the potential
for sustainable innovation in underwater systems, contributing to a resilient 
and environmentally conscious future.

% ------------------------------------------------------------ Literature Review
\section{Literature Review}
\label{sec:introduction:literature}

\gls{tpc} of reduntant menipulators is a well-estabilshed concept.
Some of the earliest works on \gls{tpc} were introduced in 
\cite{hanafusa1981}. This study proposed a method for controlling redundant 
manipulators through a \gls{tpc} control scheme, enabling the simultaneous 
execution of two tasks. Specifically, the framework was applied to a 7-DOF 
manipulator tasked with tracking a desired end-effector position while 
maintaining a constant arm posture. This approach was later expanded in 
\cite{nakamura1987}, where a more comprehensive formulation was presented, 
along with the introduction of potential functions for obstacle avoidance.

Building on the foundations laid by \cite{hanafusa1981}, \cite{nakamura1987}, 
and others, the work in \cite{siciliano1991} extended the task-priority 
framework to accommodate an arbitrary number of tasks. The proposed formulation 
employs a recursive approach, enabling efficient computation of joint 
velocities. A simulation study demonstrated the method's effectiveness in 
controlling a 7-DOF planar manipulator executing three tasks simultaneously, including 
obstacle avoidance.
Kinematic and algorithmic singularities pose distinct challenges in task-
priority control: the former can lead to excessively large joint velocities, 
while the latter may cause a breakdown in strict task prioritization. To 
address these issues, \cite{chiaverini1997} introduced the use of damped pseudo
-inverse matrices.

One notable limitation of the frameworks discussed so far is their assumption 
of decoupled kinematics and dynamics. This can be a good assumption for certain systems,
such as manipulator arms, where good tracking of desired joint velocities is possible.
However, for some systems—
particularly for \gls{uvms}s—this 
assumption does not hold. To address this limitation, \cite{khatib1987} 
proposed a framework for redundancy resolution at the dynamic level. This 
approach incorporates task dynamics, employs feedback linearization techniques, 
and utilizes dynamically consistent generalized inverses \cite{khatib1995} to 
maintain strict task priority.
However, in order to address the coupled kinematics and dynamics of a system, one has to rely heavily on an accurate
dynamic system model.
The framework was further extended in \cite{khatib2004,sentis2004}, 
allowing it to handle an arbitrary number of tasks efficiently. In these works, a 
high-DOF humanoid robot was successfully controlled in a simulation study.

Stability analysis of kinematic-level priority-based control schemes for 
redundant manipulators was conducted in \cite{antonelli2009}. Using a 
Lyapunov-based approach, the study provided sufficient conditions for stability 
based on control gains and task design. However, the proposed method assumes 
decoupled kinematics and dynamics, making it less directly applicable to \gls{uvms}s 
without additional assumptions regarding the relative speeds of
the robot's kinematics and dynamics.

A \gls{tpc} scheme specifically applied to \gls{uvms}s was presented in 
\cite{antonelli1998}. This work employed a kinematic-level \gls{tpc}
approach to control a 6-DOF \gls{auv} 
equipped with a 3-DOF planar manipulator arm. While the simulation results were 
promising, the study did not account for the system's dynamics, leaving open the question
of how well the method performs when a dynamic model is considered.

Recent research has focused on extending \gls{tpc} frameworks to incorporate
dynamic effects, addressing limitations of purely kinematic approaches.
For example, \cite{basso2020} introduced a method that uses control 
Lyapunov function-based quadratic programming to address control allocation, 
dynamic control, and redundancy resolution in redundant robotic systems. Given 
the challenges posed by disturbances and modeling inaccuracies in \gls{uvms} 
dynamics, \cite{iversflaten2022} proposed a dynamic controller based on sliding 
mode control. A simulation study demonstrated the effectiveness of this method 
in handling these challenges.
Despite recent progress, there remains a lack of research evaluating \gls{tpc}
methods for \gls{uvms}s in real-world experiments.

% --------------------------------------------------------------- Contributions
\section{Contributions}

The main contributions as presented in this thesis are as follows:
\begin{enumerate}
    \item \emph{Implementing and experimentally validating a kinemtic-level \gls{tpc} framework for the Eelume \gls{aiauv}.}
    \item \emph{Devlopment of a dynamical model and simulator for the Eelume \gls{aiauv}.}
    \item \emph{Development of a control framework interfacing with the newly developed \gls{api} for the Eelume robot.}
    \item \emph{Development of a tool showing a live visualization of the Eelume \gls{aiauv} along with desired task trajectories.}
\end{enumerate}

The first contribution is significant as it adds to the limited body of work 
on \gls{tpc} of \glspl{uvms}, demonstrating the feasibility of 
implementing such a framework on a real-world system. The remaining 
contributions are crucial as they lay the foundation for the first contribution
, as well as future work on \gls{tpc} of \glspl{uvms}. The 
development of a dynamic model and simulator allows for testing and validation 
of control strategies in a simulated environment before deployment on physical 
systems. The control framework and visualization tool enhance the usability 
and accessibility of the Eelume \gls{aiauv}, facilitating further research and 
development in this field.

% -------------------------------------------------------------- Thesis Outline

\newpage
\section{Thesis Outline}

The structure of this thesis is as follows:

\begin{itemize}
    \item \autoref{ch:background} introduces the mathematical foundations and concepts necessary for understanding the remainder of the thesis.
    \item \autoref{chap:modeling} describes how a general \gls{aiauv} can be modeled, including the coordinate frames, assumptions, and conventions used throughout this work. 
    \item \autoref{ch:tpc} presents \gls{tpc}, focusing on kinematic-level \gls{tpc}.
    \item \autoref{ch:software} presents the software developed as part of the thesis, including modules for control, visualization, and simulation.
    \item \autoref{ch:experimental_setup} introduces the experimental platform, the Eelume Model M robot, along with a discussion of system limitations launch procedure.
    \item \autoref{ch:results} presents experimental results from applying kinematic-level \gls{tpc} to the Eelume robot.
    \item Finally, \autoref{ch:conclusion} summarizes the key findings and contributions of the thesis and outlines potential directions for future work.
\end{itemize}

